\chapter{WireCenter-Tutorial}

This tutorial is a semi-automatic conversion of a PowerPoint presentation with exercises and sample problems for learning WireCenter. It was originally generated in Winter 2016 and the translation was executed in September 2017.

\section{Getting Started}
\subsection{Before you begin}

\begin{itemize}
\item  {Remarks on this tutorial: The tasks and Exercises given
  below shall provide a guided tour through the main features of
  WireCenter. Often there is not one unique way to achieve a result. The
  tutorial only poses the tasks perhaps with hints on how to solve the
  tasks.}
\item  {WireCenter was carefully developed; however, scientific
  software is not consumer software. Some functions where developed for
  a special purpose, that may not be generalized and thus not applicable
  for other
  user.}
\item  {Please report all errors and programs crashes. Please
  also report incomplete English Translation.}
\begin{itemize}
\item  {Try to reproduce the
  error}
\item  {Write down what you were doing (which function was
  called)}
\end{itemize}
\end{itemize}

\subsection{Introduction and Beginning}

\begin{itemize}
\item  {Start WireCenter, inspect the GUI}
\item  {Rotate, pan and zoom the 3D window}
\item  {Arrange the pane windows.}
\begin{itemize}
\item  {Dock the panes to the main window border}
\item  {Stack the pane windows over each other}
\end{itemize}
\item  {Change amongst the three main views (3d-view, Geometry, Report)}
\end{itemize}

\section{Basic Robot Manipulation}
\subsection{System Menu and basic robot}

\begin{itemize}
\item  {Create a new robot of the IPAnema1 design}
\item  {Manually edit the geometry (in the Geometry view)}
\item  {Use the transform feature to scale, translate, and rotate the robot frame and platform. (geometry view)}
\item  {Use model transformation during robot creation to change the robot geometry in the new robot dialog}
\item  {Use the design ribbon to apply parametric models and
  parameter changes to the
  robot}
\item  {Use design transformations to translate, scale, and
  robot the whole robot design through the ribbon
  bar}
\item  {Use the interactive mode to drag the winches}
\item  {Export a robot to wcrfx format and save it to disk;
  inspect the resulting xml file with a text
  editor}
\end{itemize}

\begin{itemize}
\item  {Load a robot from an wcrfx file}
\item  {Repeat the steps above for a planar robot with four
  cables and 1R2T motion
  pattern}
\item  {Repeat the steps above for another robot design (e.g.
  CoGiRo or
  Falcon)}
\end{itemize}

\section{Kinematic evaluation}
\subsection{Pose Inspector Pane}

\begin{itemize}
\item  {Set the platform position and orientation through the
  pane window ``pose inspector''}
\item
  {Use the interactive model to move and rotate the mobile
  platform}
\item  {Configure Pose Evaluation through the options dialog;
  add the InverseKinematics\_Standard evaluator. Observe the fields in
  the Pose Inspector
  pane}
\item  {Compute the cable length for different
  positions}
\item  {Add a suitable pose evaluator and compute possible
  cable force distributions for different
  positions}
\item  {Analyze how cable force distributions change when
  scaling or rotating the robot
  design}
\item  {Compare the results from pulley and standard model for
  inverse
  kinematics}
\end{itemize}

\begin{itemize}
\item  {Enable force visualization in the 3D-view
  settings}
\item  {Add an additional pose mapper to pose evaluation and
  change the parameterization to
  Euler-angles}
\item  {Rotate the platform and check the output on the pose
  inspector}
\end{itemize}

\section{Pose Lists}
\subsection{Pose Tab}

\begin{itemize}
\item  {Generate a pose list from an NC program and execute the
  program
  interactively}
\item  {Change the interpolation setting of the NC-interpolator
  and reload the
  program}
\item  {Write a NC program to move along a rectangle in the xy,
  xz, and yz
  plane}
\item  {Write the NC program to generate the motion described
  in the CK2013
  paper.}
\item  {Configure inverse kinematics for pose
  evaluation}
\item  {Use the export function to generate a csv
  file}
\item  {Import the data in Excel and generate plots (cable
  length over time; cable force over
  time)}
\item  {Create orientation trajectories using a NC
  program}
\end{itemize}

\begin{itemize}
\item  {Analyze change in cable force depending on
  orientation}
\item  {Use the pose options to compare different force
  distribution
  methods}
\item  {Compare pulley model with standard model along a
  trajectory}
\item  {Save the motion of the platform to a
  trajectory}
\end{itemize}

\subsection{Pose Evaluation}

\begin{itemize}
\item  {Generate a regular rectangular grid and execute pose
  evaluation}
\item  {Export the data to a csv file and open the results in
  Excel (use the ``analyze trajectory'' button on the
  ribbon)}
\item  {Compute a contour plot or a 3D-plot in Excel showing
  the
  distribution}
\item  {How many poses belong to the wrench-feasible
  workspace?}
\item  {How many poses are reachable under strict cable length
  restriction?}
\item  {Generate a random grid of poses and export the results
  to a csv
  file}
\item  {Use matlab (or similar programs) to generate a
  histogram of the cable length (do the same for the cable
  forces)}
\end{itemize}

\section{Workspace computation}

\begin{itemize}
\item  {Compute the wrench-feasible translational workspace
  using the closed-form method and the hull
  representation}
\item  {Compute the volume and surface of the wrench-feasible
  workspace; change the geometry of the robot with the parametric models
  and compare the results for workspace
  computation}
\item  {Analyze the influence of the applied wrench on the
  workspace. Apply forces in different directions; apply torques around
  different axis.
  }
\item  {Refine the workspace computation by increasing the
  number of recursive
  sub-division}
\item  {Check the computation
  time.}
\item  {Compare wrench-feasible workspace computation results
  for different force distribution methods (closed form, dykstra,
  advanced closed form,
  wrench-set)}
\item  {Change the orientation of the platform and compute the
  workspace}
\end{itemize}

\begin{itemize}
\item  {Export the workspace to a STL file and save it to
  disk}
\item  {Configure a orientation spectrum for the robot and
  compute the total orientation workspace as well as the maximum
  workspace}
\item  {Compute a cross section of the workspace in xy, xz, and
  yz
  plane}
\item  {Save the cross section to disk as svg
  file}
\item  {Compute the workspace for a regular rectangular and
  regular cylindrical
  grid}
\item  {Use a random grid for workspace
  computation}
\item  {Set the region of interest and use it for grid
  workspace}
\item  {Change the resolution for the grid workspace
  computation}
\item  {Check grid coverage for a region of
  interest}
\end{itemize}

\section{Cable interference}

\begin{itemize}
\item  {Use the cable-cable interference computation to
  determine the regions of interference for the constant orientation
  workspace}
\item  {Compute the interference region for different
  orientation of the
  platform}
\end{itemize}

\section{Visualization}

\begin{itemize}
\item  {Change the program setting for creating
  figures:}
\item  {Change between orthographic and perspective
  projection}
\item  {Set the background color to black or
  white}
\item  {Visualize the ground
  plane}
\item  {Load a CAD file into the scene (python
  command)}
\item  {Receive the platform handle, active the platform frame
  and load a CAD file to be attached to the traveling
  platform}
\item  {Delete all custom shapes (either through the ribbon or
  through
  python)}
\end{itemize}

\section{Python Scripting}
\subsection{Interactive}

\begin{itemize}
\item  {Execute a simple python command to compute the sum of
  two
  numbers}
\item  {Use the interface Iapp and Iscene to manipulate the
  WireCenter
  GUI}
\item  {Step through the four folder of the console window
  (console, python, actions,
  history)}
\item  {Select a command from the python tab and double click
  it. Configure its parameterization (if
  any)}
\item  {Select and execute a recent command from the history
  tab}
\item  {Select the edit tab and use the up/down key to execute
  recent
  commands}
\end{itemize}



\subsection{Python Scripting Script}

\begin{itemize}
\item  {Start workspace computation through a python
  script}
\item  {Change the robot geometry through a python
  script}
\item  {Perform a parameter sweep and keep track of the volume
  of the workspace.
  }
\item  {Print the pairs geometry and workspace properties on
  the
  screen}
\item  {Create a 2D plot from the results (e.g. using
  Excel)}
\end{itemize}

\subsection{Advanced data handling with scripts}

\begin{itemize}
\item  {Store the recently computed wrench-feasible workspace
  in the results
  window}
\item  {Convert the hull to a pose
  list}
\item  {Execute the pose list in the
  NC-interpolator}
\item  {Analyze the cable length for the poses on the
  hull}
\item  {Repeat the procedure for cross section
  objects}
\item  {Compare the hull for different settings of the
  workspace
  algorithm}
\item  {Compute and visualize the workspace for different force
  limits}
\end{itemize}

\begin{itemize}
\item  {Install and start a python
  development environment}
\item  {Use the scientific python package numpy, scipy, and
  matplotlib to load and display csv files created in
  WireCenter}
\end{itemize}

\subsection{Advanced Customization with Action
Scripts}

\begin{itemize}
\item  {Open WireCenter.py in a python-capable text
  editor}
\item  {Edit the print-commands in the
  default-action}
\item  {Execute the action through the WireCenter GUI (action
  on the ribbon
  bar)}
\item  {Add a custom action to the system
  script}
\item  {Check the item in the action button, inspect the system
  script using the Console action
  tab}
\end{itemize}



